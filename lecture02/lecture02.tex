% Cornell CS 2043: Unix Tools and Scripting
% Spring 2016
%
% http://cs2043-sp16.github.io/
%
% These slides have been created using the metroplis theme demo found here:
%
%     https://github.com/matze/mtheme.git
%
% I am a huge fan of the metropolis theme...thanks @matze!
\input{../common-header.input}

%%%%%%%%%%%%%%%%%%%%%%%%%%%%%%%%%%%%%%%%%%%%%%%%%%%%%%%%%%%%%%%%%%%%%%%%%%%%%%%%%%%%%%%%%%%%%%%%%%%%%%%%%%%%%%%
%%%%%%%%%%%%%%%%%%%%%%%%%%%%%%%%%%%%%%%%%%%%%%%%%%%%%%%%%%%%%%%%%%%%%%%%%%%%%%%%%%%%%%%%%%%%%%%%%%%%%%%%%%%%%%%
%%%%%%%%%%%%%%%%%%%%%%%%%%%%%%%%%%%%%%%%%%%%%%%%%%%%%%%%%%%%%%%%%%%%%%%%%%%%%%%%%%%%%%%%%%%%%%%%%%%%%%%%%%%%%%%
%%%%%%%%%%%%%%%%%%%%%%%%%%%%%%%%%%%%%%%%%%%%%%%%%%%%%%%%%%%%%%%%%%%%%%%%%%%%%%%%%%%%%%%%%%%%%%%%%%%%%%%%%%%%%%%
%
% \begin
%
%%%%%%%%%%%%%%%%%%%%%%%%%%%%%%%%%%%%%%%%%%%%%%%%%%%%%%%%%%%%%%%%%%%%%%%%%%%%%%%%%%%%%%%%%%%%%%%%%%%%%%%%%%%%%%%
%%%%%%%%%%%%%%%%%%%%%%%%%%%%%%%%%%%%%%%%%%%%%%%%%%%%%%%%%%%%%%%%%%%%%%%%%%%%%%%%%%%%%%%%%%%%%%%%%%%%%%%%%%%%%%%
%%%%%%%%%%%%%%%%%%%%%%%%%%%%%%%%%%%%%%%%%%%%%%%%%%%%%%%%%%%%%%%%%%%%%%%%%%%%%%%%%%%%%%%%%%%%%%%%%%%%%%%%%%%%%%%
%%%%%%%%%%%%%%%%%%%%%%%%%%%%%%%%%%%%%%%%%%%%%%%%%%%%%%%%%%%%%%%%%%%%%%%%%%%%%%%%%%%%%%%%%%%%%%%%%%%%%%%%%%%%%%%
\title{02 \-- The Unix File System, First Glimpse at Git}
\subtitle{CS 2043: Unix Tools and Scripting, Spring 2016 \cite{prevSemesters}}
\date{January 29th, 2016}
\author{Stephen McDowell}
\institute{Cornell University}

\begin{document}
\maketitle

% TOC
\begin{frame}{Table of contents}
  \setbeamertemplate{section in toc}[sections numbered]
  \tableofcontents[hideallsubsections]
\end{frame}

\begin{frame}{Some Logistics}
  \begin{itemize}[<+- | alert@+>]
    \item \alert<4>{HW0: You need GitHub.  And \only<4>{\textsc{\textbf{must have }}}a Unix environment.}
    \item Getting Started page updated with some videos, 32bit option available.
    \item (Poll) OH scheduling.  Thanks Jerome!
  \end{itemize}
\end{frame}

\begin{frame}[fragile]{Notation}
  Commands will be shown on slides using \texttt{teletype text}.

  \begin{block}{Introducing new commands}
    New commands will be introduced in block boxes like this one.  A summary of calling the command will
    be shown listing the command name and potential optional arguments / flags.

    \begin{center}
      some-command [opt1] [opt2]
    \end{center}
    \vspace*{1em}
  \end{block}

  To execute a command, just type its name into the shell and press return / enter.
\end{frame}

%
%%
%%%
%%%%
%%%%% section unix_filesystem_overview
%%%%%
%%%%%
%%%%%
%%%%%
\section{Unix Filesystem Overview}
\label{sec:unix_filesystem_overview}

\begin{frame}{The Unix Filesystem}
  \begin{itemize}[<+- | alert@+>]
    \item Unlike Windows, UNIX has a single global "root" directory (instead of a root directory for each disk or volume)
    \begin{itemize}[<+- | alert@+>]
      \item The root directory is just \texttt{/}
    \end{itemize}
    \item All files and directories are case sensitive
    \begin{itemize}[<+- | alert@+>]
      \item \texttt{hello.txt != hElLo.TxT}
    \end{itemize}
    \item Directories are separated by \texttt{/} instead of \texttt{\textbackslash}
    \begin{itemize}[<+- | alert@+>]
      \item UNIX: \texttt{/home/sven/lemurs}
      \item Windows: \texttt{E:\textbackslash Documents\textbackslash lemurs}
    \end{itemize}
    \item Hidden files and folders begin with a "\texttt{.}"
    \begin{itemize}[<+- | alert@+>]
      \item e.g. \texttt{.git/} (a hidden directory)
    \end{itemize}
    \item Example: my home directory
  \end{itemize}
\end{frame}

\begin{frame}{What's Where?}
  \begin{itemize}[<+- | alert@+>]
    \item \texttt{/dev}: Hardware devices, like your hard drive, USB devices
    \item \texttt{/lib}: Stores libraries, along with \texttt{/usr/lib}, \texttt{/usr/local/lib}, etc.
    \item \texttt{/mnt}: Frequently used to mount disk drives
    \item \texttt{/usr}: Mostly user-installed programs and amenities
    \item \texttt{/etc}: System-wide settings
  \end{itemize}
\end{frame}

\begin{frame}{What's Where: Programs Edition}
  Programs are usually installed in one of the "binaries" directories:
  \begin{itemize}[<+- | alert@+>]
    \item \texttt{/bin}: System programs
    \item \texttt{/usr/bin}: Most user programs
    \item \texttt{/usr/local/bin}: A few other user programs
  \end{itemize}
\end{frame}

\begin{frame}[fragile]{Personal Files}
  Your personal files are in your home directory (and its subdirectories), which is \emph{usually}\tsup{*}
  located at

  \begin{center}
    \begin{tabular}{|c|c|}
      \hline
      Linux & Mac\\ \hline
      \texttt{/home/username} & \texttt{/Users/username}\\ \hline
    \end{tabular}
  \end{center}

  There is also a built-in alias for it: \texttt{\textasciitilde}

  For example, the Desktop for the user \texttt{sven} is located at

  \begin{center}
    \begin{tabular}{|c|c|}
      \hline
      Linux & Mac\\ \hline
      \texttt{/home/sven/Desktop} & \texttt{/Users/sven/Desktop}\\ \hline
      \texttt{\textasciitilde/Desktop} & \texttt{\textasciitilde/Desktop}\\ \hline
    \end{tabular}
  \end{center}
\end{frame}
%%%%%
%%%%%
%%%%%
%%%%%
%%%%% section unix_filesystem_overview
%%%%
%%%
%%
%

%
%%
%%%
%%%%
%%%%% section basic_navigational_commands
%%%%%
%%%%%
%%%%%
%%%%%
\section{Basic Navigational Commands}
\label{sec:basic_navigational_commands}

\begin{frame}[fragile]{Where am I?}
  Most shells default to using the current path in their prompt.  If not

  \begin{block}{\colbf{P}rint \colbf{w}orking \colbf{d}irectory}
    \texttt{pwd}
    \begin{enumerate}[\--]
      \item prints the "full" path of the current directory
      \item handy on minimalist systems when you get lost
      \item can be used in scripts
    \end{enumerate}
    \vspace*{1em}
  \end{block}
\end{frame}

\begin{frame}[fragile]{What's here?}
  Knowing where you are is useful, but understanding what (or who?) else is there is too...

  \begin{block}{The \colbf{l}i\colbf{s}t command}
    \texttt{ls}
    \begin{enumerate}[\--]
      \item lists directory contents (including subdirectories)
      \item works like the dir command in Windows
      \item the \texttt{-l} flag lists detailed file / directory information (we'll learn more about flags later).
      \item use \texttt{-a} to list hidden files
    \end{enumerate}
    \vspace*{1em}
  \end{block}
\end{frame}

\begin{frame}[fragile]{Ok lets go!}
  Moving around is as easy as

  \begin{block}{\colbf{C}hanging \colbf{d}irectories}
    \texttt{cd [directory name]}
    \begin{enumerate}[\--]
      \item changes directory to \texttt{[directory name]}
      \item if not given a destination defaults to the user's home directory
      \item you can specify both absolute and relative paths
    \end{enumerate}
    \vspace*{1em}
  \end{block}

  \begin{itemize}
    \item Absolute paths start at \texttt{/}
    \begin{itemize}
      \item e.g. \texttt{cd /home/sven/Desktop}
    \end{itemize}
    \item Relative paths start at the current directory
    \begin{itemize}
      \item e.g. \texttt{cd Desktop}, if you were already at \texttt{/home/sven}
    \end{itemize}
  \end{itemize}
\end{frame}

\begin{frame}[fragile]{Relative Path Shortcuts}
  Shortcuts

  \begin{tabular}{|c|l|}
    \hline
    \texttt{\textasciitilde} & current user's home directory\\ \hline
    \texttt{.} & the current directory (this is actually useful...)\\ \hline
    \texttt{..} & the parent directory of the current directory\\ \hline
    \texttt{-} & for \texttt{cd} command, return to previous working directory\\ \hline
  \end{tabular}

  An example: starting in \texttt{/usr/local/src}

  \begin{minted}[bgcolor=bg, gobble=4]{bash}
    > cd    # now at /home/sven
    > cd -  # now at /usr/local/src
    > cd .. # now at /usr/local
  \end{minted}

\end{frame}
%%%%%
%%%%%
%%%%%
%%%%%
%%%%% section basic_navigational_commands
%%%%
%%%
%%
%

%
%%
%%%
%%%%
%%%%% section file_and_folder_manipulation
%%%%%
%%%%%
%%%%%
%%%%%
\section{File and Folder Manipulation}
\label{sec:file_and_folder_manipulation}

\begin{frame}[fragile]{Creating a new File}
  The easiest way to create an empty file is using

  \begin{block}{\colbf{touch}}
    \texttt{touch [flags] <file>}
    \begin{enumerate}[\--]
      \item adjusts the timestamp of the specified file
      \item with no flags uses the current date and time
      \item if the file does not exist, \texttt{touch} creates it
    \end{enumerate}
    \vspace*{1em}
  \end{block}

  File extensions (\texttt{.txt}, \texttt{.c}, \texttt{.py}, etc) often \textbf{don't} matter in Unix.
  Using \texttt{touch} to create a file results in a blank plain-text file (so you don't necessarily
  have to hadd \texttt{.txt} to it).
\end{frame}

\begin{frame}[fragile]{Creating a new Directory}
  No magic here...

  \begin{block}{\colbf{M}a\colbf{k}e \colbf{dir}ectory}
    \texttt{mkdir [flags] <directory1> <directory2> <...>}
    \begin{enumerate}[\--]
      \item can use relative or absolute paths
      \begin{enumerate}[\--]
        \item a.k.a. you are not restricted to making directorys in the current directory only
      \end{enumerate}
      \item need to specify at least one directory name
      \item can specify multiple, separated by spaces
    \end{enumerate}
    \vspace*{1em}
  \end{block}
\end{frame}

\begin{frame}[fragile]{File Deletion}
  Warning: once you delete a file (from the command line) there is no easy way to recover the file.

  \begin{block}{\colbf{R}e\colbf{m}ove File}
    \texttt{rm [flags] <filename>}
    \begin{enumerate}[\--]
      \item removes the file \texttt{<filename>}
      \item using wildcards (more on this later) you can remove multiple files
      \begin{enumerate}[\--]
        \item \texttt{rm *} \-- removes every file in the current directory
        \item \texttt{rm *.jpg} \-- removes every \texttt{.jpg} file in the current directory
      \end{enumerate}
      \item \texttt{rm -i <filename>} \-- prompt before deletion
    \end{enumerate}
    \vspace*{1em}
  \end{block}
\end{frame}

\begin{frame}[fragile]{Deleting Directories}
  By default, \texttt{rm} cannot remove directories. Instead we use...

  \begin{block}{\colbf{R}e\colbf{m}ove \colbf{dir}ectory}
    \texttt{rmdir [flags] <directory>}
    \begin{enumerate}[\--]
      \item removes an \textbf{empty} directory
      \item throws an error if the directory is not empty
    \end{enumerate}
    \vspace*{1em}
  \end{block}

  To delete a directory and all its subdirectories, we pass \texttt{rm} the flag \texttt{-r} (for recursive),
  e.g. \texttt{rm -r /home/sven/oldstuff}
\end{frame}

\begin{frame}[fragile]{Copy That!}
  \begin{block}{\colbf{C}o\colbf{p}y}
    \texttt{cp [flags] <file> <destination>}
    \begin{enumerate}[\--]
      \item copies from one location to another
      \item to copy multiple files, use wildcards (such as \texttt{*})
      \item to copy a complete directory use \texttt{cp -r <src> <dest>}
    \end{enumerate}
    \vspace*{1em}
  \end{block}
\end{frame}

\begin{frame}[fragile]{Move it!}
  Unlike the \texttt{cp} command, the move command automatically recurses for directories.

  \begin{block}{\colbf{M}o\colbf{v}e}
    \texttt{mv [flags] <source> <destination>}
    \begin{enumerate}[\--]
      \item moves a file or directory from one place to another
      \item also used for renaming, just move from \texttt{<oldname>} to \texttt{<newname>}
    \end{enumerate}
    \vspace*{1em}
  \end{block}
\end{frame}

\begin{frame}[fragile]{Recap}
  \begin{itemize}
    \item \texttt{ls} \-- list directory contents
    \item \texttt{cd} \-- change directory
    \item \texttt{pwd} \-- print working directory
    \item \texttt{rm} \-- remove file
    \item \texttt{rmdir} \-- remove directory
    \item \texttt{cp} \-- copy file
    \item \texttt{mv} \-- move file
  \end{itemize}
\end{frame}

%%%%%
%%%%%
%%%%%
%%%%%
%%%%% section file_and_folder_manipulation
%%%%
%%%
%%
%

%
%%
%%%
%%%%
%%%%% section flags_&_command_clarifaction
%%%%%
%%%%%
%%%%%
%%%%%
\section{Flags \& Command Clarifaction}
\label{sec:flags_&_command_clarifaction}

\begin{frame}[fragile]{Flags and Options}
  Most commands take flags and optional arguments.  These come in two general forms: switches (no argument
  required), and argument specifiers for lack of a better name.

  When specifying flags for a given command, these will modify the behavior of the command and how it
  executes.  Some flags take precedence over others, and some flags you specify can implicitly pass
  additional flags to the command.
\end{frame}

\begin{frame}[fragile]{Flags and Options: A bad Analogy}
  If you think of a command as a computer, you could think of the flags as the different
  hardware components installed.  Let's say that in this case a hard drive is a flag.

  The computer shipped to you with a CPU, motherboard, hard drive, etc and installed on that hard drive
  was the original operating system (say Windows).  When you start it, the computer was executed with the Windows flag.

  Now, you remove the original hard drive and insert another hard drive that has a different OS installed
  (say Fedora).  Then you boot your computer, only this time you ended up passing the Fedora flag.

  Nothing about the other components of the computer changed (it's just a bunch of electricity being routed around),
  but the behavior changed because of the flag you passed.
\end{frame}

\begin{frame}[fragile]{Flags and Options: Formats}
  A flag that is
  \begin{itemize}
    \item One letter is specified with a single dash (\texttt{-a})
    \item More than one letter is specified with two dashes (\texttt{--all})
  \end{itemize}
  The reason is because of how switches can be combined (next page).
\end{frame}

\begin{frame}[fragile]{Flags and Options: Switches}
  Switches take no arguments, and can be specified in a couple of different ways.  Switches are usually
  one letter, and multiple letter switches usually have a one letter alias (\texttt{--all} has the \texttt{-a} alias).
  \begin{itemize}
    \item One option:
    \begin{itemize}
      \item \texttt{ls -a}
      \item \texttt{ls --all}
    \end{itemize}
    \item Two options:
    \begin{itemize}
      \item \texttt{ls -l -Q}
    \end{itemize}
    \item Two options:
    \begin{itemize}
      \item \texttt{ls -lQ}
    \end{itemize}
    \item Applied from left to right:
    \begin{itemize}
      \item \texttt{rm -fi <file>} $\Rightarrow$ prompts
      \item \texttt{rm -if <file>} $\Rightarrow$ does \emph{not} prompt
    \end{itemize}
  \end{itemize}
\end{frame}

\begin{frame}[fragile]{Flags and Options: Argument Specifiers}
  These flags expect an input, and you will encounter two general kinds.
  \begin{itemize}
    \item The \texttt{--argument="value"} format, where the \texttt{=} and quotes are needed if \texttt{value} is
          more than one word.
    \begin{itemize}
      \item Yes: \texttt{ls --hide="Desktop" \textasciitilde/}
      \item Yes: \texttt{ls --hide=Desktop \textasciitilde/}
      \begin{itemize}
        \item one word, no quotes necessary
      \end{itemize}
      \item No: \texttt{ls --hide = "Desktop" \textasciitilde/}
      \begin{itemize}
        \item spaces by the \texttt{=} will be misinterpreted (it used \texttt{=} as the \texttt{hide} value...)
      \end{itemize}
    \end{itemize}
    \item The \texttt{--argument value} format, with a space after the \texttt{argument}. Quote rules same as above.
    \begin{itemize}
      \item \texttt{ls --hide "Desktop" \textasciitilde/}
      \item \texttt{ls --hide Desktop \textasciitilde/}
    \end{itemize}
  \end{itemize}
  {\tiny Note: The example I gave you was using the same \texttt{--hide} in both formats, but not \emph{all} commands will
    accept both.  Advise \texttt{--argument="value"} format for higher success rates.}
\end{frame}

\begin{frame}[fragile]{Flags and Options: Conventions, Warnings}
  Generally, you should always specify the flags before the arguments.  In this example, the flag is \texttt{-l} and
  \texttt{\textasciitilde/Desktop/} is the argument.
  \begin{itemize}
    \item \texttt{ls -l \textasciitilde/Desktop/} and \texttt{ls \textasciitilde/Desktop/ -l} both work
    \item there exist scenarios in which flags after arguments do \textbf{not} get processed
  \end{itemize}
  There is a special sequence \texttt{--} that signals the end of the options.  I will use another flag to demonstrate:
  \begin{itemize}
    \item \texttt{ls -l -a \textasciitilde/Desktop/} $\Rightarrow$ executes as expected
    \item \texttt{ls -l -- -a \textasciitilde/Desktop/} $\Rightarrow$ only used \texttt{-l}
    \begin{itemize}
      \item {\tiny "\texttt{ls: cannot access -a: No such file or directory}"}
      \item \texttt{-a} was treated as an \emph{argument}, and there is no \texttt{-a} directory (for me)
    \end{itemize}
  \end{itemize}
\end{frame}

\begin{frame}[fragile]{Flags and Options: Conventions, Warnings (cont)}
  The special sequence \texttt{--} that signals the end of the options is often most useful if you need to do something
  special.  Suppose I wanted to make the folder \texttt{-a} on my Desktop.

  \begin{minted}[bgcolor=bg, gobble=4]{bash}
    > cd ~/Desktop # for demonstration purpose
    > mkdir -a     # fails: invalid option -- 'a'
    > mkdir -- -a  # success! (ls to confirm)
    > rmdir -a     # fails: invalid option -- 'a'
    > rmdir -- -a  # success! (ls to confirm)
  \end{minted}

  This trick can be useful in \emph{many} scenarios, and generally arises when you need to work with special characters
  of some sort.
\end{frame}

\begin{frame}[fragile]{Your new best friend}
  How do I know what the flags / options for all of these commands are?

  \begin{block}{The \colbf{man}ual command}
    \texttt{man <command\_name>}
    \begin{enumerate}[\--]
      \item Loads the manual (manpage) for the specified command
      \item Unlike google, manpages are \textbf{system-specific}
      \item Usually very comprehensive.  Sometimes \emph{too} comprehensive
      \item Type \texttt{/<keyword>}
      \item The \texttt{n} key jumps through the search results
    \end{enumerate}
  \end{block}
  Search example on next page if that was confusing.  Intended for side-by-side follow-along.
\end{frame}

\begin{frame}[fragile]{Man oh man}
  \begin{minted}[bgcolor=bg, gobble=4]{bash}
    > man man  #  you now have the manual loaded
    > /useful  #  type /useful, then hit enter
    ############ [first result highlighted]
    > n        #  followed by enter
    ############ [next result highlighted]
  \end{minted}
  Note that there are subtle differences between options on different systems.  For example, \texttt{ls -B}:
  \begin{itemize}
    \item BSD/OSX: Force printing of non-printable characters in file names as \texttt{\textbackslash xxx},
          where \texttt{xxx} is the numeric value of the character in \emph{octal}.
    \item Fedora, Ubuntu: do not list implied entries ending with \texttt{\textasciitilde}
    \begin{itemize}
      \item In these OS's, files ending with \texttt{\textasciitilde} are \emph{temporary} backup files that
            certain programs generate
    \end{itemize}
  \end{itemize}
\end{frame}

%%%%%
%%%%%
%%%%%
%%%%%
%%%%% section flags_&_command_clarifaction
%%%%
%%%
%%
%

\begin{frame}[allowframebreaks]{References}
  \bibliography{references}
  \bibliographystyle{abbrv}
\end{frame}

\end{document}
