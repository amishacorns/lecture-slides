% Cornell CS 2043: Unix Tools and Scripting
% Spring 2016
%
% http://cs2043-sp16.github.io/
%
% These slides have been created using the metroplis theme demo found here:
%
%     https://github.com/matze/mtheme.git
%
% I am a huge fan of the metropolis theme...thanks @matze!
\input{../common-header.input}

%%%%%%%%%%%%%%%%%%%%%%%%%%%%%%%%%%%%%%%%%%%%%%%%%%%%%%%%%%%%%%%%%%%%%%%%%%%%%%%%%%%%%%%%%%%%%%%%%%%%%%%%%%%%%%%
%%%%%%%%%%%%%%%%%%%%%%%%%%%%%%%%%%%%%%%%%%%%%%%%%%%%%%%%%%%%%%%%%%%%%%%%%%%%%%%%%%%%%%%%%%%%%%%%%%%%%%%%%%%%%%%
%%%%%%%%%%%%%%%%%%%%%%%%%%%%%%%%%%%%%%%%%%%%%%%%%%%%%%%%%%%%%%%%%%%%%%%%%%%%%%%%%%%%%%%%%%%%%%%%%%%%%%%%%%%%%%%
%%%%%%%%%%%%%%%%%%%%%%%%%%%%%%%%%%%%%%%%%%%%%%%%%%%%%%%%%%%%%%%%%%%%%%%%%%%%%%%%%%%%%%%%%%%%%%%%%%%%%%%%%%%%%%%
%
% \begin
%
%%%%%%%%%%%%%%%%%%%%%%%%%%%%%%%%%%%%%%%%%%%%%%%%%%%%%%%%%%%%%%%%%%%%%%%%%%%%%%%%%%%%%%%%%%%%%%%%%%%%%%%%%%%%%%%
%%%%%%%%%%%%%%%%%%%%%%%%%%%%%%%%%%%%%%%%%%%%%%%%%%%%%%%%%%%%%%%%%%%%%%%%%%%%%%%%%%%%%%%%%%%%%%%%%%%%%%%%%%%%%%%
%%%%%%%%%%%%%%%%%%%%%%%%%%%%%%%%%%%%%%%%%%%%%%%%%%%%%%%%%%%%%%%%%%%%%%%%%%%%%%%%%%%%%%%%%%%%%%%%%%%%%%%%%%%%%%%
%%%%%%%%%%%%%%%%%%%%%%%%%%%%%%%%%%%%%%%%%%%%%%%%%%%%%%%%%%%%%%%%%%%%%%%%%%%%%%%%%%%%%%%%%%%%%%%%%%%%%%%%%%%%%%%
\title{03 \-- Manipulating Files and Using Git}
\subtitle{CS 2043: Unix Tools and Scripting, Spring 2016 \cite{prevSemesters}}
\date{February 1st, 2016}
\author{Stephen McDowell}
\institute{Cornell University}

\begin{document}
\maketitle

% TOC
\begin{frame}{Table of contents}
  \setbeamertemplate{section in toc}[sections numbered]
  \tableofcontents[hideallsubsections]
\end{frame}

\begin{frame}{Some Logistics}
  \begin{itemize}[<+- | alert@+>]
    \item Last day to add is Wednesday 2/3.
    \item HW0: Due today at 5pm
    \item My OH are Tuesdays 6:00pm - 7:00pm, Gates G19
    \item On moving forward independently, and the usage of \texttt{sudo}
    \begin{itemize}[<+- | alert@+>]
      \item I strongly advise taking a \emph{snapshot} of your VM
    \end{itemize}
    \item A note about HW1...
  \end{itemize}
\end{frame}

%
%%
%%%
%%%%
%%%%% section working_with_files
%%%%%
%%%%%
%%%%%
%%%%%
\section{Working with Files}
\label{sec:working_with_files}

\begin{frame}[fragile]{Users and Groups}
  Like most OS's, Unix allows multiple people to use the same machine at once.  The question: who has access
  to what?

  \begin{itemize}[<+- | alert@+>]
    \item Access to files depends on the users' account
    \item All accounts are presided over by the \colbf{S}uper\colbf{u}ser, or \texttt{root} account
    \item Each user has absolute control over any files they own, which can only be superseded by \texttt{root}
    \item Files can also be owned by a \texttt{group}, allowing more users to have access
  \end{itemize}
\end{frame}

\begin{frame}[fragile]{File Ownership}
  You can discern who owns a file many ways, the most immediate being \texttt{ls -l}\wl

  \begin{block}{Permissions with \texttt{ls}}
    {\small
    \texttt{> ls -l Makefile}\\
    \texttt{-rw-rw-r--. 1 sven users 4.9K Jan 31 04:42 Makefile}\\
    \texttt{\hphantom{-rw-rw-r--. 1 }\colbf{sven}\hphantom{ users }\# the user}\\
    \texttt{\hphantom{-rw-rw-r--. 1 sven }\colbf{users }\# the group}\\
    }
  \end{block}

  The third column is the \emph{user}, and the fourth column is the \emph{group}.
\end{frame}

\begin{frame}[fragile]{What is this RWX Nonsense?}
  R = read, W = write, X = execute\wl
  \begin{itemize}[<+- | alert@+>]
    \item[] \hspace*{-4ex}-\alert<2>{rwx}\alert<3>{rwx}\alert<4>{rwx}
    \item User permissions
    \item Group permissions
    \item Other permissions (a.k.a. neither the owner, nor a member of the group)
    \item[] \hspace*{-4ex}Directory permissions begin with a \texttt{d} instead of a \texttt{-}.
  \end{itemize}

\end{frame}

\begin{frame}[fragile]{An example}
  \begin{itemize}[<+- | alert@+>]
    \item[] \hspace*{-4ex}What would the permissions \texttt{-rwxr-----} mean?\wl
    \item It is a file.
    \item User can read and write to the file, as well as execute it
    \item Group members are allowed to read the file, but cannot write to or execute
    \item Other cannot do \emph{anything} with it
  \end{itemize}
\end{frame}

\begin{frame}[fragile]{Changing Permissions}
  \begin{block}{\colbf{Ch}ange \colbf{Mod}e}
    \texttt{chmod <mode> <file>}
    \begin{itemize}
      \item Changes file / directory permissions to \texttt{<mode>}
      \item The format of \texttt{<mode>} is a combination of three fields:
      \begin{itemize}
        \item Who is affected \-- a combination of \texttt{u}, \texttt{g}, \texttt{o}, or \texttt{a} (all)
        \item Whether adding or removing permissions; add with \texttt{+}, remove with \texttt{-}
        \item Which permissions are being modified \-- any combination of \texttt{r}, \texttt{w}, \texttt{x}
      \end{itemize}
      \item Or you can specify mode in octal: user, then group, then other
      \begin{itemize}
        \item e.g. \texttt{777} means user=7, group=7, other=7
      \end{itemize}
    \end{itemize}
  \end{block}
  The octal version can be confusing, but will save you time.  Excellent resource in \cite{chmod}.
\end{frame}

\begin{frame}[fragile]{Changing Ownership}
  Changing the group
  \begin{block}{\colbf{Ch}ange \colbf{Gr}ou\colbf{p}}
    \texttt{chgrp group <file>}
    \begin{itemize}
      \item Changes the group ownership of \texttt{<file>}
    \end{itemize}
  \end{block}

  As the super user, you can change who owns a file

  \begin{block}{\colbf{Ch}ange \colbf{Own}ership}
    \texttt{chown user:group <file>}
    \begin{itemize}
      \item Changes the ownership of \texttt{<file>}
      \item \texttt{group} is optional
      \item the \texttt{-R} flag is useful for recursively modifying everything in a directory
    \end{itemize}
  \end{block}
\end{frame}

\begin{frame}[fragile]{File Ownership, Alternate}
  If you are like me, you often forget which column is which in \texttt{ls -l}...\wl

  \begin{block}{\colbf{Stat}us of a file or filesystem}
    \texttt{stat [opts] <filename>}
    \begin{itemize}
      \item Gives you a wealth of information, generally more than you will need
      \item \texttt{Uid} is the user, \texttt{Gid} is the group
      \item Can be useful if you want to mimic file permissions you don't know
      \begin{itemize}
        \item \texttt{--format=\%A}: human readable, e.g. \texttt{-rw-rw-r--}
        \item \texttt{--format=\%a}: octal (great for \texttt{chmod}), e.g. \texttt{664}
      \end{itemize}
    \end{itemize}
  \end{block}
\end{frame}

\begin{frame}[fragile]{Platform Notes I}
  Convenience flag for \texttt{chown} and \texttt{chmod} on non-BSD Unix

  \begin{minted}[bgcolor=bg, gobble=4]{bash}
    > chmod --reference=<src> <dest>
  \end{minted}

  It will set the permissions of \texttt{dest} to the permissions of \texttt{src}!
  Mac users: sorry :/

  The \texttt{stat} on BSD: the \texttt{--format} does not exist, it is just \texttt{-f}.  The options seem
  to be the same, but read the man page.
\end{frame}

\begin{frame}[fragile]{Platform Notes II}
  The \texttt{stat} command performs a little differently on OSX by default.  For example, on the \texttt{Makefile}
  it produces this giant wall (on one line, continued for presentation purposes):

  \begin{minted}[bgcolor=bg, gobble=4, fontsize=\footnotesize]{bash}
    > stat Makefile
    > 16777218 6517959 -rw-r--r-- 1 sven staff 0 4945
        "Feb  1 11:48:14 2016" "Jan 31 07:02:42 2016"
        "Jan 31 08:28:22 2016" "Jan 31 07:02:42 2016"
        4096 16 0 Makefile
  \end{minted}

  To get more useful output for the intended purpose of \texttt{stat} in how I am presenting it, you need to do
  \texttt{stat -x Makefile}.  This will print out the \texttt{Uid} and \texttt{Gid} for you.
\end{frame}

%%%%%
%%%%%
%%%%%
%%%%%
%%%%% section working_with_files
%%%%
%%%
%%
%

%
%%
%%%
%%%%
%%%%% section types_of_files
%%%%%
%%%%%
%%%%%
%%%%%
\section{Types of Files and Usages}
\label{sec:types_of_files}

\begin{frame}[fragile]{Plain Files}
  Plain text files are human-readable, and are usually used for things like
  \begin{itemize}[<+- | alert@+>]
    \item Documentation
    \item Application settings
    \item Source code
    \item Logs
    \item Anything you may want to read via the terminal (e.g. README.txt)
  \end{itemize}
\end{frame}

\begin{frame}[fragile]{Binary Files}
  Binary files are not human-readable.  They are written in the language your computer prefers.
  \begin{itemize}[<+- | alert@+>]
    \item Executables
    \item Libraries
    \item Media files
    \item Archives (.zip, etc)
  \end{itemize}
\end{frame}

\begin{frame}[fragile]{Reading Files Without Opening}
  \begin{block}{Print a file to the screen}
    \texttt{cat <filename>}
    \begin{itemize}
      \item Prints the contents of the file to the terminal window
    \end{itemize}
    \texttt{cat <file1> <file2>}
    \begin{itemize}
      \item Prints \texttt{file1} first, then \texttt{file2}.
    \end{itemize}
  \end{block}
  \begin{block}{\colbf{more}}
    \texttt{more <filename>}
    \begin{itemize}
      \item Scroll through one page at a time
    \end{itemize}
  \end{block}
  \begin{block}{\colbf{less}}
    \texttt{less <filename>}
    \begin{itemize}
      \item Scroll by pages or lines (mouse wheel, space bar, and arrows)
    \end{itemize}
  \end{block}
\end{frame}

\begin{frame}[fragile]{Beginning and End}
  Long files can be a pain with the previous tools.

  \begin{block}{\colbf{Head} and \colbf{Tail}}
    \texttt{head -[numlines] <filename>}\\
    \texttt{tail -[numlines] <filename>}
    \begin{itemize}
      \item Prints the first / last numlines of the file
      \item Default is 10 lines
    \end{itemize}
  \end{block}
\end{frame}

\begin{frame}[fragile]{Not Really a File...YET}
  You can talk to yourself in the terminal too!

  \begin{block}{\colbf{Echo}}
    \texttt{echo <text>}
    \begin{itemize}
      \item Prints the input string to the standard output (the terminal)
      \item We will soon learn how to use \texttt{echo} to put things into files, append to files, etc
    \end{itemize}
  \end{block}
\end{frame}
%%%%%
%%%%%
%%%%%
%%%%%
%%%%% section types_of_files
%%%%
%%%
%%
%

%
%%
%%%
%%%%
%%%%% section let_s_git_started
%%%%%
%%%%%
%%%%%
%%%%%
\section{Let's Git Started}
\label{sec:let_s_git_started}

\begin{frame}[fragile]{Another Brief Git Demo}
  If you are not at lecture, don't worry about this slide not making any sense.\wl

  \begin{minted}[bgcolor=bg, gobble=4]{bash}
    > git clone <url>
    > git status
    > git add <file(s)>
    > git commit
    > git push
  \end{minted}
\end{frame}

%%%%%
%%%%%
%%%%%
%%%%%
%%%%% section let_s_git_started
%%%%
%%%
%%
%

%
%%
%%%
%%%%
%%%%% section demo_time_
%%%%%
%%%%%
%%%%%
%%%%%
\section{Demo Time!}
\label{sec:demo_time_}
\begin{frame}[fragile]{Our first in class demo}
  Instructions are here:\wl

  {\small \href{https://github.com/cs2043-sp16/lecture-demos/tree/master/lec03}{https://github.com/cs2043-sp16/lecture-demos/tree/master/lec03}}
\end{frame}
%%%%%
%%%%%
%%%%%
%%%%%
%%%%% section demo_time_
%%%%
%%%
%%
%

\begin{frame}[allowframebreaks]{References}
  \bibliography{references}
  \bibliographystyle{abbrv}
\end{frame}

\end{document}
